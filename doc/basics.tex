% !TEX root = SoLoud.tex
%===============================================================================

\chapter{Core: Basics}

\section{SoLoud::Soloud Object}

In order to use SoLoud, you have to create a SoLoud::Soloud object. The object must be cleaned up or destroyed before your back-end is shut down; the safest way to do this is to call soloud.deinit() manually before terminating.

The object may be global, member variable, or even a local variable, it can be allocated from the heap or the stack, as long as the above demand is met. If the back-end gets destroyed before the back-end clean-up call is made, the result is undefined. As in, bad. Most likely, a crash. Bluescreens in Windows are not out of the question.

\begin{lstlisting}{frame=single, framerule=1pt}
SoLoud::Soloud *soloud = new SoLoud::Soloud; // object created
SoLoud::sdl_init(soloud);   // back-end initialization
...
soloud->deinit();           // clean-up
delete soloud;              // this cleans up too
\end{lstlisting}

Seriously: remember to call the cleanup function. The SoLoud object destructor also calls the cleanup function, but if you perform your application's teardown in an unpredictable order (such as having the SoLoud object be a global variable), the back-end may end up trying to use resources that are no longer available. So, it's best to call the cleanup function manually.

\section{Soloud.play()}

The play function can be used to start playing a sound source. The function has more than one parameter, with typical default values set to most of them.

\begin{lstlisting}{frame=single, framerule=1pt}
int play(AudioSource &aSound, 
         float aVolume = 1.0f, // Full volume 
         float aPan = 0.0f,    // Centered
         int aPaused = 0,      // Not paused
         int aBus = 0);        // Primary bus
\end{lstlisting}

The play function returns a channel handle which can be used to adjust the parameters of the sound while it's playing. The most common parameters can be set with the play function parameters, but for more complex processing you may want to start the sound paused, adjust the parameters, and then un-pause it.

\begin{lstlisting}{frame=single, framerule=1pt}
int h = soloud.play(sound, 1, 0, 1);  // start paused
soloud.setRelativePlaySpeed(h, 0.8f); // change a parameter
soloud.setPause(h, 0);                // unpause
\end{lstlisting}

\section{Soloud.seek()}

You can seek to a specific time in the sound with the seek function. Note that the seek operation may be rather heavy, and some audio sources will not support seeking backwards at all.

\begin{lstlisting}{frame=single, framerule=1pt}
int h = soloud.play(sound, 1, 0, 1); // start paused
soloud.seek(h, 3.8f);                // seek to 3.8 seconds
soloud.setPause(h, 0);               // unpause
\end{lstlisting}

\section{Soloud.stop()}

The stop function can be used to stop a sound.

\begin{lstlisting}{frame=single, framerule=1pt}
soloud.stop(h); // Silence!
\end{lstlisting}

\section{Soloud.stopAll()}

The stop function can be used to stop all sounds. Note that this will also stop the protected sounds.

\begin{lstlisting}{frame=single, framerule=1pt}
soloud.stopAll(); // Total silence!
\end{lstlisting}

\section{Soloud.stopSound()}

The stop function can be used to stop all sounds that were started through a certain sound source. Will also stop protected sounds.

\begin{lstlisting}{frame=single, framerule=1pt}
soloud.stopSound(duck); // silence all the ducks
\end{lstlisting}

\section{Soloud.setGlobalVolume() / Soloud.getGlobalVolume()}

These functions can be used to get and set the global volume. The volume is applied before clipping. Lowering the global volume is one way to combat clipping artifacts.

\begin{lstlisting}{frame=single, framerule=1pt}
float v = soloud.getGlobalVolume(); // get the current global volume
soloud.setGlobalVolume(v * 0.5f);   // halve it
\end{lstlisting}

Note that the volume is not limited to 0..1 range. Negative values may result in strange behavior, while huge values will likely cause distortion.

\section{Soloud.setPostClipScaler() / Soloud.getPostClipScaler()}

These functions can be used to get and set the post-clip scaler. The scaler is applied after clipping. Sometimes lowering the post-clip result sound volume may be beneficial. For instance, recording video with some video capture software results in distorted sound if the volume is too high.

\begin{lstlisting}{frame=single, framerule=1pt}
float v = soloud.getPostClipScaler(); // get the current post-clip scaler
soloud.setPostClipScaler(v * 0.5f);   // halve it
\end{lstlisting}

Note that the scale is not limited to 0..1 range. Negative values may result in strange behavior, while huge values will likely cause distortion.