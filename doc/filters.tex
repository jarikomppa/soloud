% !TEX root = SoLoud.tex
%===============================================================================

\chapter{SoLoud::Filter}

Filters can be used to modify the sound some way. Typical uses for a filter are to create environmental effects, like echo, or to modify the way the speech synthesizer sounds like.

Like audio sources, filters are implemented with two classes; Filter and FilterInstance. These are, however, typically much simpler than those derived from the AudioSource and AudioInstance classes.

\section{Filter}
\begin{lstlisting}{frame=single, framerule=1pt}
class Example : public Filter
{
public:
  virtual FilterInstance *createInstance();
};
\end{lstlisting}

As with audio sources, the only required function is the createInstance().

\section{FilterInstance}

\begin{lstlisting}{frame=single, framerule=1pt}
class ExampleInstance : public FilterInstance
{
public:
  virtual void filter(
    float *aBuffer,     int aSamples, 
    int aChannels,      float aSamplerate, 
    float aTime);
    
  virtual void filterChannel(
    float *aBuffer,     int aSamples, 
    float aSamplerate,  float aTime, 
    int aChannel,       int aChannels);
    
  virtual float getFilterParameter(
    int aAttributeId);
    
  virtual void setFilterParameter(
    int aAttributeId,   float aValue);
    
  virtual void fadeFilterParameter(
    int aAttributeId,   float aTo, 
    float aTime,        float aStartTime);
    
  virtual void oscillateFilterParameter(
    int aAttributeId,   float aFrom, 
    float aTo,          float aTime, 
    float aStartTime);
};
\end{lstlisting}
The filter instance has no mandatory functions, but you may want to implement either filter() or filterChannel() to do some actual work.

\section{FilterInstance.filter()}

The filter() function is the main workhorse of a filter. It gets a buffer of samples, channel count, samplerate and current stream time, and is expected to overwrite the samples with filtered ones.

If channel count is not one, the layout of the buffer is such that the first channel's samples come first, followed by the second channel's samples, etc.

So if dealing with stereo samples, aBuffer first has aSamples floats for the first channel, followed by aSamples floats for the second channel.

The default implementation calls filterChannel for every channel in the buffer.

\section{FilterInstance.filterChannel()}

Most filters are simpler to write on a channel-by-channel basis, so that they only deal with mono samples. In this case, you may want to use the filterChannel() function instead. The default implementation of filter() calls the filterChannel() for every channel in the source.

\section{FilterInstance.getFilterParameter()}

This function is needed to support the changing of live filter parameters. The default implementation returns zero. The attribute number is defined by the filter.

\section{FilterInstance.setFilterParameter()}

This function is needed to support the changing of live filter parameters. The default implementation does nothing. The attribute number is defined by the filter.

\section{FilterInstance.fadeFilterParameter()}

This function is needed to support the changing of live filter parameters. The default implementation does nothing. The attribute number is defined by the filter.

\section{FilterInstance.oscillateFitlerParameter()}

This function is needed to support the changing of live filter parameters. The default implementation does nothing. The attribute number is defined by the filter.




